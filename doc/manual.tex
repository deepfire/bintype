\documentclass[%
	%draft,
	%submission,
	%compressed,
	final,
	%
	%technote,
	%internal,
	%submitted,
	%inpress,
	%reprint,
	%
	%titlepage,
	notitlepage,
	%anonymous,
	narroweqnarray,
	inline,
	twoside,
        %invited,
	]{ieee}

\newcommand{\latexiie}{\LaTeX2{\Large$_\varepsilon$}}

%\usepackage{ieeetsp}	% if you want the "trans. sig. pro." style
%\usepackage{ieeetc}	% if you want the "trans. comp." style
%\usepackage{ieeeimtc}	% if you want the IMTC conference style

% Use the `endfloat' package to move figures and tables to the end
% of the paper. Useful for `submission' mode.
%\usepackage {endfloat}

% Use the `times' package to use Helvetica and Times-Roman fonts
% instead of the standard Computer Modern fonts. Useful for the 
% IEEE Computer Society transactions.
%\usepackage{times}
% (Note: If you have the commercial package `mathtime,' (from 
% y&y (http://www.yandy.com), it is much better, but the `times' 
% package works too). So, if you have it...
%\usepackage {mathtime}

% for any plug-in code... insert it here. For example, the CDC style...
%\usepackage{ieeecdc}

\begin{document}

%----------------------------------------------------------------------
% Title Information, Abstract and Keywords
%----------------------------------------------------------------------
\title[BINTYPE Manual]{%
       BINTYPE Manual}

% format author this way for journal articles.
% MAKE SURE THERE ARE NO SPACES BEFORE A \member OR \authorinfo
% COMMAND (this also means `don't break the line before these
% commands).
\author[KOSYREV\'{A}R]{Serge Kosyrev\authorinfo{S.\,Kosyrev is employed by Elvees, Moscow,
	Russia.
       E-mail: \_deepfire@feelingofgreen.ru}%
}
\author[GROMOFF\'{A}R]{Samium Gromoff\authorinfo{S.\,Gromoff is employed by Elvees, Moscow,
	Russia.
       E-mail: \_deepfire@feelingofgreen.ru}%
}

%\journal{IEEE Trans.\ on Instrum.\ Meas.}
%\titletext{, VOL.\ 46, NO.\ 6, DECEMBER\ 1997}
%\ieeecopyright{0018--9456/97\$10.00 \copyright\ 1997 IEEE}
%\lognumber{xxxxxxx}
%\pubitemident{S 0018--9456(97)09426--6}
%\loginfo{Manuscript received September 27, 1997.}
%\firstpage{1217}

%\confplacedate{Ottawa, Canada, May 19--21, 1997}

\maketitle               

\begin{abstract} 
The impulse behind this project was born out of the perception,
that writing procedural parser algorithms for binary formats
is boring, and, as the serialization patterns underlying them are
largely boilerplate, basically avoidable in a vast number of cases.
This manual describes a Common Lisp-based domain-specific language for binary
format specifications, along with a tool, to automatically transform such
specifications into fully-fledged parsers.
It is available at \mbox{git://git.feelingofgreen.ru/bintype}.
\end{abstract}

\begin{keywords}
Binary formats, serialization, specification-driven parsing.
\end{keywords}

%----------------------------------------------------------------------
% SECTION I: Introduction
%----------------------------------------------------------------------
\section{Introduction}

\PARstartCal Whenever the problem of data interchange poses itself,
whether due to the needs of persistence or replication of computation states, 


%----------------------------------------------------------------------
% SECTION ...: Conclusions
%----------------------------------------------------------------------
\section{Conclusion}

The BINTYPE tool, described here, is provided on an ``as-is'' basis at

\begin{verbatim}
  git://git.feelingofgreen.ru/bintype
\end{verbatim}

\noindent along with its dependency

\begin{verbatim}
  git://git.feelingofgreen.ru/pergamum
\end{verbatim}

\noindent Any bug report is welcome, especially if accompanied by the solution!
Reports should be made, via e-mail, to the author.

%----------------------------------------------------------------------
% The bibliography. This bibliography was generated using the following
% two lines:
%\bibliographystyle{IEEEbib}
%\bibliography{ieeecls}
% where, the contents of the ieeecls.bib file was:
%
%@book{lamport,
%        AUTHOR = "Leslie Lamport",
%         TITLE = "A Document Preparation System: {\LaTeX} User's Guide
%                  and Reference Manual",
%       EDITION = "Second",
%     PUBLISHER = "Addison-Wesley",
%       ADDRESS = "Reading, MA",
%          YEAR = 1994,
%          NOTE = "Be sure to get the updated version for \LaTeX2e!"
%}
%
%@book{goossens,
%        AUTHOR = "Michel Goossens and Frank Mittelbach and
%                  Alexander Samarin",
%         TITLE = "The {\LaTeX} Companion",
%     PUBLISHER = "Addison-Wesley",
%       ADDRESS = "Reading, MA",
%          YEAR = 1994,
%}
%
% The ieeecls.bbl file was manually included here to make the distribution
% of this paper easier. You need not do it for your own papers.

\begin{thebibliography}{1}

\bibitem{fjeld}
Frode Vatvedt Fjeld, {\em Binary-types},
\newblock http://www.cs.uit.no/~frodef/sw/binary-types/; 2003

\bibitem{seibel}
Peter Seibel, {\em Practical Common Lisp}, ch. 24
\newblock Apress, 2005. ISBN13: 978-1-59059-239-7

\bibitem{clx}
CLX developers, {\em CLX}, buffer.lisp, gl.lisp, macros.lisp
\newblock http://www.cliki.net/CLX, 1987-2007

\end{thebibliography}

\end{document}
